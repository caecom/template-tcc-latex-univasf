%--------------------------------------------------------------------------------
% Constrói a capa com base na seção de identificação do main.tex
%--------------------------------------------------------------------------------
\begin{capa}
    \setlength{\belowcaptionskip}{0pt}
    \setlength{\abovecaptionskip}{0pt}
    \setlength{\intextsep}{-18pt}
        \begin{figure}[h]
        \begin{center}
            \includegraphics[scale=1.0]{img/LOGO_UNIVASF_big.pdf}
        \end{center}
      \end{figure}

        %\includegraphics[scale=0.6]{img/univasf.jpg}
        \center
      {\ABNTEXchapterfont\large\imprimirinstituicao}

      \vspace*{2cm}
          {\imprimirautor}
      \vspace*{2cm}
        \begin{center}
        \ABNTEXchapterfont\bfseries\large\imprimirtitulo
        \end{center}
      \vfill

      \ABNTEXchapterfont\bfseries\large\imprimirlocal\\
      \the\year

      \vspace*{1cm}
\end{capa}
%--------------------------------------------------------------------------------
% Constrói a folha de rosto com base na seção de identificação do main.tex
%--------------------------------------------------------------------------------
\begin{folhaderosto}
    \center
      {\ABNTEXchapterfont\large\imprimirinstituicao}

    \vspace*{2cm}
          {\imprimirautor}
      \vspace*{2cm}
    \vspace*{\fill}

    {\ABNTEXchapterfont\bfseries\large\imprimirtitulo}
    \vspace*{\fill}

    {\hspace{.45\textwidth}
    \begin{minipage}{.5\textwidth}
      \SingleSpacing
      \imprimirpreambulo \\ \\

      {\imprimirorientadorRotulo~\imprimirorientador\par}
      {\imprimircoorientadorRotulo~\imprimircoorientador\par}

    \end{minipage}%
    \vspace*{\fill}}%
    \vspace*{\fill}
      \ABNTEXchapterfont\bfseries\large\imprimirlocal\\
      \the\year
    \vspace*{1cm}
\end{folhaderosto}

%--------------------------------------------------------------------------------
% Constrói a ficha catalográfia com base na seção de identificação do main.tex
% Está comentado porque no final das contas a biblioteca do seu campus que gera a
% numeração, você pode adicionar os numeros aqui, ou anexar o pdf gerado por eles
% ao documento.
%--------------------------------------------------------------------------------
%\begin{fichacatalografica}
%	\vspace*{\fill}					% Posição vertical
%	\hrule							% Linha horizontal
%	\begin{center}					% Minipage Centralizado
%	\begin{minipage}[c]{12.5cm}		% Largura
%
%	\imprimirautor
%
%	\hspace{0.5cm} \imprimirtitulo  / \imprimirautor. --
%	\imprimirlocal, \the\year-
%
%	\hspace{0.5cm} xx p. : il. (algumas color.) ; 30 cm.\\
%
%	\hspace{0.5cm} \imprimirorientadorRotulo~\imprimirorientador\\
%
%	\hspace{0.5cm}
%	\parbox[t]{\textwidth}{\imprimirtipotrabalho~--~\imprimirinstituicao,
%	\the\year.}\\
%
%	\hspace{0.5cm}
%		1. Palavra-chave1.
%		2. Palavra-chave2.
%		I. Orientador.
%		II. Universidade xxx.
%		III. Faculdade de xxx.
%		IV. Título\\
%
%	\hspace{8.75cm} CDU 02:141:005.7\\
%
%	\end{minipage}
%	\end{center}
%	\hrule
%\end{fichacatalografica}

%--------------------------------------------------------------------------------
% Anexando a ficha catalogáfica e a folha de aprovação
%--------------------------------------------------------------------------------
\includepdf[pages=-]{anexos/ficha.pdf}

\includepdf[pages=-]{anexos/aprovacao.pdf}

\setlength{\ABNTEXsignwidth}{12cm}

%--------------------------------------------------------------------------------
% Está comentado pelo mesmo motivo da ficha catalográfica
%--------------------------------------------------------------------------------
%\begin{folhadeaprovacao}
%	\begin{center}
%	    {\ABNTEXchapterfont\bfseries\large\imprimirinstituicao}
%	    \vspace*{\fill}
%
%	    {\ABNTEXchapterfont\bfseries\large FOLHA DE APROVAÇÃO}
%	    \vspace*{\fill}
%
%	    {\ABNTEXchapterfont\bfseries\large\imprimirautor}
%
%	    \vspace*{\fill}\vspace*{\fill}
%	    {\ABNTEXchapterfont\bfseries\large\imprimirtitulo}
%	    \vspace*{\fill}
%
%	    {\hspace{.45\textwidth}
%		\begin{minipage}{.5\textwidth}
%			\SingleSpacing
%			\ABNTEXchapterfont\imprimirpreambulo \\ \\
%
%			{\ABNTEXchapterfont\imprimirorientadorRotulo~\imprimirorientador\par}
%			{\ABNTEXchapterfont\imprimircoorientadorRotulo~\imprimircoorientador\par}
%
%		\end{minipage}%
%	    \vspace*{\fill}}
%	\end{center}
%
%	\vspace*{\fill}
%
%	\begin{center}
%			 \ABNTEXchapterfont\large Aprovado em: \_\_\_\_ de \_\_\_\_ de 2017
%	\end{center}

%	\vspace*{\fill}

%	\begin{center}
%			 \ABNTEXchapterfont\bfseries\large Banca Examinadora
%	\end{center}
%
%   \ABNTEXchapterfont\assinatura{Fábio Nelson de Sousa Pereira, Mestre, Universidade Federal do Vale do São Francisco}
%	\ABNTEXchapterfont\assinatura{Jorge Luis Cavalcanti Ramos, Doutor, Universidade Federal do vale do São Francisco}
%  \ABNTEXchapterfont\assinatura{Ricardo Argenton Ramos, Doutor, Universidade Federal do Vale do São Francisco}
%	 \vspace*{\fill}


%\end{folhadeaprovacao}

%--------------------------------------------------------------------------------
% Insere a epígrafe
%--------------------------------------------------------------------------------
\newpage
\vspace*{\fill}
\begin{flushright}
    \textit{Lorem Ipsum...}
\end{flushright}

%--------------------------------------------------------------------------------
% Seção de agradecimentos
%--------------------------------------------------------------------------------
\begin{agradecimentos}

\lipsum[2-4]

\end{agradecimentos}

%--------------------------------------------------------------------------------
% Insere a segunda epígrafe
%--------------------------------------------------------------------------------
\begin{epigrafe}
  \vspace*{\fill}
  \begin{flushright}
    Se pude enxergar a tão grande distância, foi subindo nos ombros de gigantes.\\
     \vspace{\baselineskip}
    \textbf{Isaac Newton}\\
    \textbf{Carta à Robert Hooke, 1676}
  \end{flushright}
\end{epigrafe}



%--------------------------------------------------------------------------------
% Seção de resumos
%--------------------------------------------------------------------------------
% resumo em português
\setlength{\absparsep}{18pt} % ajusta o espaçamento dos parágrafos do resumo
\begin{resumo}


  \textbf{Palavras-chave}: \textit{Palavra em inglês 1, Palavra em inglês 2, Palavra em inglês 3, Palavra em inglês 4}, Palavra 5.

\end{resumo}
\newpage

% resumo em inglês
\begin{resumo}[Abstract]
\begin{otherlanguage*}{english}

  \vspace{\onelineskip}

  \noindent
  \textbf{Key-words}: \textit{Palavra em inglês 1, Palavra em inglês 2, Palavra em inglês 3, Palavra em inglês 4, Palavra em inglês 4}.

\end{otherlanguage*}
\end{resumo}


%---------------------------------------------------------------------------------
% Insere lista de ilustrações
%---------------------------------------------------------------------------------
\begin{KeepFromToc} % Este comando evita que todas as seções dentro dele de apareçam no sumário
\pdfbookmark[0]{\listfigurename}{lof}
\listoffigures
\cleardoublepage


%---------------------------------------------------------------------------------
% Insere lista de tabelas
%---------------------------------------------------------------------------------
% \pdfbookmark[0]{\listtablename}{lot}
% \listoftables
% \cleardoublepage

%---------------------------------------------------------------------------------
% Insere lista de quadros
%---------------------------------------------------------------------------------
% \pdfbookmark[0]{\listofquadrosname}{loq}
% \listofquadros*
% \cleardoublepage

%---------------------------------------------------------------------------------
% Ajusta lista de código - alterar de figures para códigos - by @Gabrielr2508
%---------------------------------------------------------------------------------
% \makeatletter
% \let\l@listing\l@figure
% \def\newfloat@listoflisting@hook{\let\figurename\listingname}
% \makeatother

%---------------------------------------------------------------------------------
% Insere lista de códigos - by @leolleocomp
%---------------------------------------------------------------------------------
% \listoflistings

\end{KeepFromToc}

%---------------------------------------------------------------------------------
% Insere lista de abreviaturas e siglas
%---------------------------------------------------------------------------------
\begin{siglas}
  \item[LI]       Lorem Ipsum
    \item[LII]		Lorem Ipsum Ipsum

\end{siglas}

%---------------------------------------------------------------------------------
% Insere o sumario
%---------------------------------------------------------------------------------
\pdfbookmark[0]{\contentsname}{toc}
\tableofcontents*
\cleardoublepage


