%--------------------------------------------------------------------------------------
% Este arquivo contém a sua funtamentação teórica
%--------------------------------------------------------------------------------------
\chapter{Referencial teórico}

\section{Seção de exemplo 1 - Como fazer citações}

\lipsum[1-20]


\section{Seção de exemplo 2 - Como inserir figuras}



%--------------------------------------------------------------------------------------
% Esse é um exemplo de figura composta de outras subfiguras
%--------------------------------------------------------------------------------------
\begin{figure}[!htb]
\centering
    \caption{\label{img:figura1} Exemplo de figura composta}
    \subcaptionbox{\label{img:subfigura1} Subfigura 1}{\includegraphics[scale=.1]{img/placeholder}}\qquad
    \subcaptionbox{\label{img:subfigura2} Subfigura 2}{\includegraphics[scale=.1]{img/placeholder}}
    \legend{\textbf{Fonte:} \citeonline{SUA-REFERENCIA}}
\label{fig:dag}
\end{figure}

%\begin{figure}[!htb]
%\centering
%    \caption{\label{img:telas} Telas da aplicação cliente}
%    \subcaptionbox{\label{img:inicial} Abertura}{\includegraphics[scale=.12]{img/APP/inicial}}\qquad
%    \subcaptionbox{\label{img:login} \textit{Login}}{\includegraphics[scale=.12]{img/APP/login}}\qquad
%    \subcaptionbox{\label{img:cadastro} Cadastro}{\includegraphics[scale=.12]{img/APP/cadastro}}\qquad
%    \subcaptionbox{\label{img:hist-rel}Sobre}{\includegraphics[scale=.12]{img/APP/sobre}}\\
%    \vspace{1.5em}
%    \subcaptionbox{\label{img:dados_atuais}Dados atuais}{\includegraphics[scale=.15]{img/APP/atual}}\qquad
%    \subcaptionbox{\label{img:hist-time}Seleção de período}{\includegraphics[scale=.15]{img/APP/periodo}}\qquad
%    \subcaptionbox{\label{img:hist-rel}Exibir histórico}{\includegraphics[scale=.15]{img/APP/historico}}\\
%    \vspace{2.5em}
%    \legend{\textbf{Fonte:} O Autor}
%\label{fig:dag}
%\end{figure}

%--------------------------------------------------------------------------------------
% Esse é um exemplo de figura simples
%--------------------------------------------------------------------------------------
\imagem{0.25}{placeholder}{Uma figura simples}{O autor}


\newpage % Quebra a página

Para referenciar uma imagem deve ser usada a tag REF, como exemplo, estamos referenciando a figura \ref{img:placeholder}. Isso vale tanto para figuras simples quanto para as compostas, como por exemplo as figuras \ref{img:subfigura1} e \ref{img:subfigura2}. Ao inserir uma figura, ela é automaticamente identificada e incluída no elemento pré-textual da lista de figuras.





\subsection{Subseção de exemplo 1 - Sobre as seções}



\section{Seção de exemplo 3 - Sobre tabelas}


\begin{table}[!htb]
	\centering
	\caption{\label{tab:crossplatform} Tipos de aplicações e abordagens preferenciais.}
	\begin{adjustbox}{max width=\textwidth}
		\begin{tabular}{@{} p{5cm} ccc @{}}
		\toprule
		\textbf{Código da Aplicação} & \textbf{Web} & \textbf{Híbrida} & \textbf{Interpretada / Compilação Cruzada} \\ \hline

		\textbf{Aplicações baseadas em dados providos por um servidor} &
			3 & 2 & 1
		\\ \hline

		\textbf{Aplicações independentes} & 1 & 2 & 3\\ \hline

		\textbf{Aplicações baseadas em sensores e processamento de dados no dispositivo} & 1 & 2 & 3\\ \hline

		\textbf{Aplicações baseadas em sensores e processamento de dados no servidor} & 1 & 3 & 2\\ \hline

		\textbf{Aplicações Cliente-Servidor} & 1 & 3 & 2 \\ \bottomrule
	\end{tabular}
	\end{adjustbox}
	\legend{\textbf{Fonte:} \citeonline{raj2012study} (Traduzido)}
\end{table}


