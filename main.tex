\documentclass[
	12pt,				% tamanho da fonte
	openany,			% capítulos começam em pág ímpar (insere página vazia caso preciso)
	oneside, 			% oneside - twoside
	a4paper,			% tamanho do papel.
	chapter=TITLE,		% títulos de capítulos convertidos em letras maiúsculas
	section=TITLE,		% títulos de seções convertidos em letras maiúsculas
	sumario=tradicional,	
	%subsection=TITLE,	% títulos de subseções convertidos em letras maiúsculas
	%subsubsection=TITLE,% títulos de subsubseções convertidos em letras maiúsculas
	english,			% idioma adicional para hifenização
	brazil,				% o último idioma é o principal do documento
	]{abntex2}

% ---------------------------------------------------------------------------
% Inclui os comandos do projeto
% ---------------------------------------------------------------------------
% -----------------------------------------------------------------------------
% Pacotes fundamentais
% -----------------------------------------------------------------------------
\usepackage{xcolor}
\newcommand\myworries[1]{\textcolor{red}{[#1]}}
\usepackage{lmodern}		% Usa a fonte Latin Modern (Serifada, tipo Times New Roman
%\usepackage{helvet}		% Usa a fonte Helvetica (Tipo Arial)
%\renewcommand{\familydefault}{\sfdefault} tira o serifado
\usepackage[T1]{fontenc}		% Selecao de codigos de fonte.
\usepackage[utf8]{inputenc}		% Codificacao do documento (conversão automática dos acentos)
\usepackage{indentfirst}		% Indenta o primeiro parágrafo de cada seção.
\usepackage{color}				% Controle das cores
\usepackage{tikz}				% Inclusão de gráficos
\usepackage{graphicx}			% Inclusão de gráficos
\usepackage{microtype} 			% para melhorias de justificação
% -----------------------------------------------------------------------------
% Pacotes adicionais, usados no anexo do modelo de folha de identificação
% -----------------------------------------------------------------------------
\usepackage{multicol}
\usepackage{multirow}
% -----------------------------------------------------------------------------
% Pacotes adicionais, usados apenas no âmbito do Modelo Canônico do abnteX2
% -----------------------------------------------------------------------------
\usepackage{lipsum}				% para geração de dummy text
% -----------------------------------------------------------------------------
% Pacotes de citações
% -----------------------------------------------------------------------------
\usepackage[brazilian,hyperpageref]{backref}	 % Paginas com as citações na bibliografia
\usepackage[alf,abnt-etal-list=3,abnt-etal-cite=2, abnt-emphasize=bf,abnt-repeated-author-omit=yes,abnt-etal-text=emph]{abntex2cite}	% Citações padrão ABNT
\usepackage{pdflscape}
\usepackage{footnote}
\usepackage{pdfpages}
\usepackage{caption}

% -----------------------------------------------------------------------------
% Pacotes adicionados por @leolleocomp
% -----------------------------------------------------------------------------
\usepackage{booktabs}
\usepackage{adjustbox}
\usepackage{subcaption}
\usepackage[labelfont=bf]{caption}
\usepackage{gensymb}
\usepackage{amsmath}
\usepackage{array}
% \usepackage{float}
\usepackage{xcolor,colortbl}
\usepackage{longtable}
\usepackage{scalefnt}
\usepackage{listings}			% inserir codigo fonte
\usepackage{morewrites} % necessário pois estamos screvendo muitos arquivos
\usepackage{enumitem}

% -----------------------------------------------------------------------------
% Pacotes adicionados por @Gabrielr2508
% -----------------------------------------------------------------------------
\usepackage{hyperref}

\usepackage{tocloft}
% -- permite a adição de células especiais em tabelas
\newcommand{\specialcell}[2][c]{%
  \begin{tabular}[#1]{@{}c@{}}#2\end{tabular}}

\newcounter{equationset}
\newcommand{\equationset}[1]{% \equationset{<caption>}
  \refstepcounter{equationset}% Step counter
  \noindent\makebox[\linewidth]{Equação ~\theequationset: #1}
 }

%-------------------------------------------------------------------------------
% Adequação dos títulos dos capitulos, seções, subseções às normas da Univasf
% Added by @Gabrielr2508
%-------------------------------------------------------------------------------
\renewcommand{\ABNTEXchapterfont}{\fontseries{b}}
\renewcommand{\ABNTEXchapterfontsize}{\normalsize}

\renewcommand{\ABNTEXsectionfont}{\fontseries{m}}
\renewcommand{\ABNTEXsectionfontsize}{\normalsize}

\renewcommand{\ABNTEXsubsectionfont}{\fontseries{b}}
\renewcommand{\ABNTEXsubsectionfontsize}{\normalsize}

\renewcommand{\ABNTEXsubsubsectionfont}{\fontseries{m}}
\renewcommand{\ABNTEXsubsubsectionfontsize}{\normalsize}

%-------------------------------------------------------------------------------
% CONFIGURAÇÕES DE PACOTES
% Configurações do pacote backref
%-------------------------------------------------------------------------------
% Usado sem a opção hyperpageref de backref
\renewcommand{\backrefpagesname}{Citado na(s) página(s):~}
% Texto padrão antes do número das páginas
\renewcommand{\backref}{}
% Define os textos da citação
\renewcommand*{\backrefalt}[4]{
  \ifcase #1 %
    %Nenhuma citação no texto.%
  \or
    Citado na página #2.%
  \else
    Citado #1 vezes nas páginas #2.%
  \fi}%

%-------------------------------------------------------------------------------
% Configurações de aparência do PDF final
%-------------------------------------------------------------------------------
% alterando o aspecto da cor azul
\definecolor{blue}{RGB}{41,5,195}

% informações do PDF
\makeatletter
\hypersetup{
       %pagebackref=true,
    pdftitle={\@title},
    pdfauthor={\@author},
      pdfsubject={\imprimirpreambulo},
      pdfcreator={LaTeX with abnTeX2},
    pdfkeywords={abnt}{latex}{abntex}{abntex2}{relatório técnico},
    colorlinks=true,			% false: boxed links; true: colored links
      linkcolor=black,				% color of internal links
      citecolor=black,				% color of links to bibliography
      filecolor=black,			% color of file links
    urlcolor=black,
    bookmarksdepth=4
}
\makeatother
% ---

% ---
% Espaçamentos entre linhas e parágrafos
% ---

% O tamanho do parágrafo é dado por:
\setlength{\parindent}{1.3cm}

% Controle do espaçamento entre um parágrafo e outro:
\setlength{\parskip}{0.2cm}  % tente também \onelineskip

%-------------------------------------------------------------------------------
% compila o indice
%-------------------------------------------------------------------------------
\makeindex
% ---

%-------------------------------------------------------------------------------
% Comando para inserir imagens de forma simples
%-------------------------------------------------------------------------------
\newcommand{\imagem}[4]
{%			\imagem{x.x}{nomeimg}{titulo}{fonte}
  \begin{figure}[!htb]
    \caption{\label{img:#2}#3}
    \begin{center}
      \includegraphics[scale=#1]{img/#2}
    \end{center}
        \legend{\textbf{Fonte:} #4}
  \end{figure}
}%

%-------------------------------------------------------------------------------
% Creio que esses comandos sejam para desenhar algo, aguardando explicações de @leolleocomp
%-------------------------------------------------------------------------------
\newcommand{\xx} {$\bigotimes$}
\newcommand{\oo} {$\bigcirc$}

%-------------------------------------------------------------------------------
% Biblioteca para códigos-fonte
%-------------------------------------------------------------------------------
\usepackage[newfloat=true]{minted}

%-------------------------------------------------------------------------------
% Caixas batutas - by @leolleocomp
%-------------------------------------------------------------------------------
\usepackage[most]{tcolorbox}
\tcbuselibrary{breakable}

\tcbuselibrary{minted}
\tcbset{listing engine=minted}

\definecolor{bg}{rgb}{0.95,0.95,0.95}

\SetupFloatingEnvironment{listing}{name=Código, listname=Lista de códigos}

%-------------------------------------------------------------------------------
% configuração do contador dos códigos-fonte - by @leolleocomp
% assim como as figuras, começa em 1
\newcounter{sourcecode}
%-------------------------------------------------------------------------------
%-------------------------------------------------------------------------------
% @leolleocomp
% stackoverflow code
% peguei da resposta abaixo
% https://stackoverflow.com/questions/24086366/change-latex-minted-listings-numbering-to-include-current-section?answertab=votes#tab-top
%-------------------------------------------------------------------------------
\makeatletter
\renewcommand*{\thelisting}{\thesourcecode}
\makeatother

%-------------------------------------------------------------------------------
% Peçam explicações a @leolleo
% WHO DID THIS?
%-------------------------------------------------------------------------------
\newcommand{\Ididthis}{
%	\legend{\textbf{Fonte:} O autor (\the\year).}
\legend{\textbf{Fonte:} O autor}
}

\newcommand{\Otherguydidthis}[1]{
  \legend{\textbf{Fonte:} \citeonline{#1}.}
}

%-------------------------------------------------------------------------------
% Comando para inserir códigos - by @leolleocomp
%-------------------------------------------------------------------------------
\newcommand{\sourcecode}[4]{
\begin{listing}[h]
  \refstepcounter{sourcecode}
  \caption{#1}
  \label{cmd:#2}
  \inputminted[linenos, bgcolor=bg, tabsize=4,breaklines]{#3}{codes/#4}
  \Ididthis
\end{listing}
}


% -----------------------------------------------------------------------------
% Pacotes adicionados por @ruanmed
% -----------------------------------------------------------------------------
\usepackage[binary-units=true]{siunitx}


%-------------------------------------------------------------------------------
% Comando para inserir códigos - by @ruanmed
%-------------------------------------------------------------------------------
\usepackage{caption}

\newenvironment{code}{\captionsetup{type=listing}}{}
\SetupFloatingEnvironment{listing}{name=Código}

\newcommand{\sourcecodenolist}[4]{
  \begin{code}
      \refstepcounter{sourcecode}
        \captionof{listing}{#1 }
        \label{code:#2}
        \inputminted[linenos, bgcolor=bg, tabsize=4,breaklines]{#3}{codes/#4}
        \Wedidthis
    \end{code}
}

\newcommand{\sourcecodeinline}[2]{
  \mintinline[linenos, bgcolor=bg, tabsize=4,breaklines]{#1}{#2}
}


% CONFIGURACAO DO SUMARIO
%-------------------------------------------------------------------------------
% Modifica o espaçamento no sumário
% Nao ha espacos, exceto para as entradas de capitulos
\setlength{\cftbeforeparagraphskip}{0pt}
\setlength{\cftbeforesubsectionskip}{0pt}
\setlength{\cftbeforesectionskip}{0pt}
\setlength{\cftbeforesubsubsectionskip}{0pt}
\setlength{\cftbeforechapterskip}{\onelineskip}

% Alteração da indentação dos itens do sumário
\cftsetindents{chapter}{0pt}{42pt}
\cftsetindents{section}{0pt}{42pt}
\cftsetindents{subsection}{0pt}{42pt}
\cftsetindents{subsubsection}{0pt}{42pt}

% Modifica a formatacao dos textos

% Secao Primaria (Chapter): Caixa alta, Negrito, tamanho 12
\makeatletter
\settocpreprocessor{chapter}{%
  \let\tempf@rtoc\f@rtoc%
  \def\f@rtoc{%
  \texorpdfstring{\bfseries\MakeTextUppercase{\tempf@rtoc}}{\tempf@rtoc}}%
}
\makeatother

% Novo list of (listings) para QUADROS
% retirado de https://github.com/abntex/abntex2/wiki/HowToCriarNovoAmbienteListing

\newcommand{\quadroname}{Quadro}
\newcommand{\listofquadrosname}{Lista de quadros}

\newfloat[chapter]{quadro}{loq}{\quadroname}
\newlistof{listofquadros}{loq}{\listofquadrosname}
\newlistentry{quadro}{loq}{0}

% configurações para atender às regras da ABNT
\setfloatadjustment{quadro}{\centering}
\counterwithout{quadro}{chapter}
\renewcommand{\cftquadroname}{\quadroname\space}
\renewcommand*{\cftquadroaftersnum}{\hfill--\hfill}

% Configuração de posicionamento padrão:
\setfloatlocations{quadro}{hbtp}

% Trocar o título da lista de ilustrações para lista de figuras
\addto\captionsbrazil{
  \renewcommand{\listfigurename}
    {Lista de figuras}
}


% ---------------------------------------------------------------------------
% IDENTIFICAÇÃO
% ---------------------------------------------------------------------------
\titulo{\MakeUppercase{Estudo Comparativo de métodos para classificação de magneto-encefalograma entre indivíduos}}
\autor{RICARDO VALÉRIO TEIXEIRA DE MEDEIROS SILVA}
\local{JUAZEIRO - BA}
\orientador{Prof. Dr. Rosalvo Ferreira de Oliveira Neto}
%\coorientador{Prof. Dr Rodrigo Pereira Ramos} %talvez...
\instituicao{
UNIVERSIDADE FEDERAL DO VALE DO SÃO FRANCISCO
	\par
CURSO DE GRADUAÇÃO EM ENGENHARIA DE COMPUTAÇÃO}
\tipotrabalho{Trabalho de Conclusão de Curso}
\preambulo{Trabalho apresentado à Universidade Federal do Vale do São Francisco - Univasf, Campus Juazeiro, como requisito da obtenção do título de Bacharel em Engenharia de Computação.}


% -----------------------------------------------------------------------------
% CONFIGURACAO DO SUMARIO - by @Gabrielr2508
% Precisa estar aqui, por isso não foi para o commands.tex, não descobrimos o motivo, %caso saiba, por favor, faça um pull request! :D
% -----------------------------------------------------------------------------
% Secao primaria (Chapter) Caixa alta, Negrito, tamanho 12
\makeatletter
\renewcommand*{\l@chapter}[2]{%
  \l@chapapp{\uppercase{#1}}{#2}{\cftchaptername}}
\makeatother
% Secao secundaria (Section) Caixa baixa, Negrito, tamanho 12
\renewcommand{\cftsectionfont}{\uppercase} %ponha \rmfamily se quiser serifadas...

% Secao terciaria (Subsection) Caixa baixa, negrito, tamanho 12
\renewcommand{\cftsubsectionfont}{\bfseries}

% Secao quaternaria (Subsubsection) Caixa baixa, tamanho 12
\renewcommand{\cftsubsubsectionfont}{\normalfont}

% Seção quinaria (subsubsubsection) Caixa baixa, sem negrito, tamanho 12
\renewcommand{\cftparagraphfont}{\normalfont\itshape}

% -----------------------------------------------------------------------------
% Início do TCC 
% -----------------------------------------------------------------------------
\begin{document}

	\frenchspacing % Retira espaço extra obsoleto entre as frases.
	
	\pretextual
		%--------------------------------------------------------------------------------
% Constrói a capa com base na seção de identificação do main.tex
%--------------------------------------------------------------------------------
\begin{capa}
    \setlength{\belowcaptionskip}{0pt}
    \setlength{\abovecaptionskip}{0pt}
    \setlength{\intextsep}{-18pt}
        \begin{figure}[h]
        \begin{center}
            \includegraphics[scale=1.0]{img/LOGO_UNIVASF_big.pdf}
        \end{center}
      \end{figure}

        %\includegraphics[scale=0.6]{img/univasf.jpg}
        \center
      {\ABNTEXchapterfont\large\imprimirinstituicao}

      \vspace*{2cm}
          {\imprimirautor}
      \vspace*{2cm}
        \begin{center}
        \ABNTEXchapterfont\bfseries\large\imprimirtitulo
        \end{center}
      \vfill

      \ABNTEXchapterfont\bfseries\large\imprimirlocal\\
      \the\year

      \vspace*{1cm}
\end{capa}
%--------------------------------------------------------------------------------
% Constrói a folha de rosto com base na seção de identificação do main.tex
%--------------------------------------------------------------------------------
\begin{folhaderosto}
    \center
      {\ABNTEXchapterfont\large\imprimirinstituicao}

    \vspace*{2cm}
          {\imprimirautor}
      \vspace*{2cm}
    \vspace*{\fill}

    {\ABNTEXchapterfont\bfseries\large\imprimirtitulo}
    \vspace*{\fill}

    {\hspace{.45\textwidth}
    \begin{minipage}{.5\textwidth}
      \SingleSpacing
      \imprimirpreambulo \\ \\

      {\imprimirorientadorRotulo~\imprimirorientador\par}
      {\imprimircoorientadorRotulo~\imprimircoorientador\par}

    \end{minipage}%
    \vspace*{\fill}}%
    \vspace*{\fill}
      \ABNTEXchapterfont\bfseries\large\imprimirlocal\\
      \the\year
    \vspace*{1cm}
\end{folhaderosto}

%--------------------------------------------------------------------------------
% Constrói a ficha catalográfia com base na seção de identificação do main.tex
% Está comentado porque no final das contas a biblioteca do seu campus que gera a
% numeração, você pode adicionar os numeros aqui, ou anexar o pdf gerado por eles
% ao documento.
%--------------------------------------------------------------------------------
%\begin{fichacatalografica}
%	\vspace*{\fill}					% Posição vertical
%	\hrule							% Linha horizontal
%	\begin{center}					% Minipage Centralizado
%	\begin{minipage}[c]{12.5cm}		% Largura
%
%	\imprimirautor
%
%	\hspace{0.5cm} \imprimirtitulo  / \imprimirautor. --
%	\imprimirlocal, \the\year-
%
%	\hspace{0.5cm} xx p. : il. (algumas color.) ; 30 cm.\\
%
%	\hspace{0.5cm} \imprimirorientadorRotulo~\imprimirorientador\\
%
%	\hspace{0.5cm}
%	\parbox[t]{\textwidth}{\imprimirtipotrabalho~--~\imprimirinstituicao,
%	\the\year.}\\
%
%	\hspace{0.5cm}
%		1. Palavra-chave1.
%		2. Palavra-chave2.
%		I. Orientador.
%		II. Universidade xxx.
%		III. Faculdade de xxx.
%		IV. Título\\
%
%	\hspace{8.75cm} CDU 02:141:005.7\\
%
%	\end{minipage}
%	\end{center}
%	\hrule
%\end{fichacatalografica}

%--------------------------------------------------------------------------------
% Anexando a ficha catalogáfica e a folha de aprovação
%--------------------------------------------------------------------------------
\includepdf[pages=-]{anexos/ficha.pdf}

\includepdf[pages=-]{anexos/aprovacao.pdf}

\setlength{\ABNTEXsignwidth}{12cm}

%--------------------------------------------------------------------------------
% Está comentado pelo mesmo motivo da ficha catalográfica
%--------------------------------------------------------------------------------
%\begin{folhadeaprovacao}
%	\begin{center}
%	    {\ABNTEXchapterfont\bfseries\large\imprimirinstituicao}
%	    \vspace*{\fill}
%
%	    {\ABNTEXchapterfont\bfseries\large FOLHA DE APROVAÇÃO}
%	    \vspace*{\fill}
%
%	    {\ABNTEXchapterfont\bfseries\large\imprimirautor}
%
%	    \vspace*{\fill}\vspace*{\fill}
%	    {\ABNTEXchapterfont\bfseries\large\imprimirtitulo}
%	    \vspace*{\fill}
%
%	    {\hspace{.45\textwidth}
%		\begin{minipage}{.5\textwidth}
%			\SingleSpacing
%			\ABNTEXchapterfont\imprimirpreambulo \\ \\
%
%			{\ABNTEXchapterfont\imprimirorientadorRotulo~\imprimirorientador\par}
%			{\ABNTEXchapterfont\imprimircoorientadorRotulo~\imprimircoorientador\par}
%
%		\end{minipage}%
%	    \vspace*{\fill}}
%	\end{center}
%
%	\vspace*{\fill}
%
%	\begin{center}
%			 \ABNTEXchapterfont\large Aprovado em: \_\_\_\_ de \_\_\_\_ de 2017
%	\end{center}

%	\vspace*{\fill}

%	\begin{center}
%			 \ABNTEXchapterfont\bfseries\large Banca Examinadora
%	\end{center}
%
%   \ABNTEXchapterfont\assinatura{Fábio Nelson de Sousa Pereira, Mestre, Universidade Federal do Vale do São Francisco}
%	\ABNTEXchapterfont\assinatura{Jorge Luis Cavalcanti Ramos, Doutor, Universidade Federal do vale do São Francisco}
%  \ABNTEXchapterfont\assinatura{Ricardo Argenton Ramos, Doutor, Universidade Federal do Vale do São Francisco}
%	 \vspace*{\fill}


%\end{folhadeaprovacao}

%--------------------------------------------------------------------------------
% Insere a epígrafe
%--------------------------------------------------------------------------------
\newpage
\vspace*{\fill}
\begin{flushright}
    \textit{Lorem Ipsum...}
\end{flushright}

%--------------------------------------------------------------------------------
% Seção de agradecimentos
%--------------------------------------------------------------------------------
\begin{agradecimentos}

\lipsum[2-4]

\end{agradecimentos}

%--------------------------------------------------------------------------------
% Insere a segunda epígrafe
%--------------------------------------------------------------------------------
\begin{epigrafe}
  \vspace*{\fill}
  \begin{flushright}
    Se pude enxergar a tão grande distância, foi subindo nos ombros de gigantes.\\
     \vspace{\baselineskip}
    \textbf{Isaac Newton}\\
    \textbf{Carta à Robert Hooke, 1676}
  \end{flushright}
\end{epigrafe}



%--------------------------------------------------------------------------------
% Seção de resumos
%--------------------------------------------------------------------------------
% resumo em português
\setlength{\absparsep}{18pt} % ajusta o espaçamento dos parágrafos do resumo
\begin{resumo}


  \textbf{Palavras-chave}: \textit{Palavra em inglês 1, Palavra em inglês 2, Palavra em inglês 3, Palavra em inglês 4}, Palavra 5.

\end{resumo}
\newpage

% resumo em inglês
\begin{resumo}[Abstract]
\begin{otherlanguage*}{english}

  \vspace{\onelineskip}

  \noindent
  \textbf{Key-words}: \textit{Palavra em inglês 1, Palavra em inglês 2, Palavra em inglês 3, Palavra em inglês 4, Palavra em inglês 4}.

\end{otherlanguage*}
\end{resumo}


%---------------------------------------------------------------------------------
% Insere lista de ilustrações
%---------------------------------------------------------------------------------
\begin{KeepFromToc} % Este comando evita que todas as seções dentro dele de apareçam no sumário
\pdfbookmark[0]{\listfigurename}{lof}
\listoffigures
\cleardoublepage


%---------------------------------------------------------------------------------
% Insere lista de tabelas
%---------------------------------------------------------------------------------
% \pdfbookmark[0]{\listtablename}{lot}
% \listoftables
% \cleardoublepage

%---------------------------------------------------------------------------------
% Insere lista de quadros
%---------------------------------------------------------------------------------
% \pdfbookmark[0]{\listofquadrosname}{loq}
% \listofquadros*
% \cleardoublepage

%---------------------------------------------------------------------------------
% Ajusta lista de código - alterar de figures para códigos - by @Gabrielr2508
%---------------------------------------------------------------------------------
% \makeatletter
% \let\l@listing\l@figure
% \def\newfloat@listoflisting@hook{\let\figurename\listingname}
% \makeatother

%---------------------------------------------------------------------------------
% Insere lista de códigos - by @leolleocomp
%---------------------------------------------------------------------------------
% \listoflistings

\end{KeepFromToc}

%---------------------------------------------------------------------------------
% Insere lista de abreviaturas e siglas
%---------------------------------------------------------------------------------
\begin{siglas}
  \item[LI]       Lorem Ipsum
    \item[LII]		Lorem Ipsum Ipsum

\end{siglas}

%---------------------------------------------------------------------------------
% Insere o sumario
%---------------------------------------------------------------------------------
\pdfbookmark[0]{\contentsname}{toc}
\tableofcontents*
\cleardoublepage




	\textual
		\pagestyle{simple}
		%--------------------------------------------------------------------------------------
% Este arquivo contém a sua introdução, objetivos e organização do trabalho
%--------------------------------------------------------------------------------------
\chapter{Introdução}

O desafio de levar educação e formação profissional a lugares remotos, onde,
dificilmente, a formação presencial tradicional conseguiria alcançar de maneira
efetiva, é uma das principais bandeiras da Educação a Distância (EAD).
Entretanto, outros desafios surgem em decorrência da expansão da modalidade:
como garantir a qualidade dessa formação? como transpor modelos de educação
presencial para a distância? como atuar de maneira a prevenir e reduzir os altos
índices de evasão, ainda, verificados na modalidade? São questões como essas que
devem ser respondidas a partir do desenvolvimento de pesquisas nessa modalidade.
O uso de novas tecnologias e de novos processos pode contribuir com essas
pesquisas.

%-------------------suprimido temporariamente -------------------------------------------------------
% A partir do uso de algoritmos de aprendizagem de máquina já consolidados na área
% de mineração de dados, pretende-se testar modelos preditivos da evasão de alunos
% na EAD, que foram testados em alunos e cursos de outra instituição, para
% avaliar a sua aplicabilidade nos cursos de graduação a distância
% ofertados pela Universidade Federal do Vale do São Francisco (UNIVASF).

%-------------------suprimido temporariamente \section{Justificativa}

Diversas iniciativas reforçam o crescimento da EAD e exigem uma atenção maior
nos aspectos importantes para a consolidação e manutenção das atividades dessa
modalidade nas instituições. Dentre esses aspectos, está a necessidade de
pesquisas voltadas para a modalidade, como forma de se agregar procedimentos
validados cientificamente, ferramentas de gestão mais eficientes e metodologias
inovadoras, capazes de superar grandes desafios impostos pela EAD.

O avanço da modalidade de EAD requer o desenvolvimento de recursos que permitam
o acompanhamento de cursos oferecidos em um ambiente virtual de aprendizagem.
Esses recursos podem ser obtidos a partir de metodologias de análise que
envolvem o conhecimento das estratégias pedagógicas dos cursos EAD, o
levantamento das necessidades apontadas pelos profissionais que atuam na área e
na elicitação dos requisitos para implementação de ferramentas de visualização
de dados de diversas atividades dentro de um contexto educacional.
\cite{ramos2016abordagem}

Com a expansão do EAD de maneira responsável e planejada, com infraestrutura
compatível e recursos humanos qualificados, será possível a oferta de novos
cursos pelas instituições, disseminando conhecimento e possibilitando mais
oportunidades para o desenvolvimento regional.

Para \citeonline{ramos2016abordagem}, os altos índices de evasão dos alunos em
cursos EAD representam um grande desafio para todos os que atuam na modalidade.
Além desses índices estarem em níveis elevados, observou-se também que estão em
crescimento. Com isso há uma necessidade contínua de desenvolvimento de
pesquisas que apontem caminhos, métodos e ferramentas que os auxiliem a
enfrentar melhor esse problema. O uso de técnicas estatísticas e de mineração de
dados, em conjunto com teorias consolidadas na modalidade, pode fundamentar
modelos eficientes de detecção precoce do risco de evasão pelos alunos.

No estudo apresentado por \citeonline{ramos2016abordagem}, foram desenvolvidos,
testados e validados, modelos preditivos da evasão de estudantes de graduação em
cursos ofertados na modalidade EAD, tomando como base as variáveis que compõem
cada um dos construtos da Teoria da Distância Transacional
\cite{moore2008teoria}. Essa pesquisa ocorreu a partir dos dados de cursos de
licenciatura em Biologia e Pedagogia, ambos ofertados por EAD, na Universidade
de Pernambuco (UPE).

A citada pesquisa testou cinco algoritmos de classificação para definição dos
modelos preditivos: Árvore de Decisão, Máquina de Vetor de Suporte (SVM, do inglês, \textit{Support Vector Machine}), Rede
Neural Artificial, K-Vizinhos Mais Próximos (KNN, do inglês, \textit{K-nearest
Neighbors}) e Regressão Logística, sendo este último o que apresentou resultados
mais relevantes, embora os demais não ficaram muito distantes, nas métricas
analisadas.

A partir dessa referência, este estudo foi desenvolvido no sentido de verificar
se o mesmo conjunto de variáveis usadas e os algoritmos de classificação
aplicados, podem também ser replicados e validados em outro cenário educacional.
Desta vez nos cursos de graduação em Administração Pública e na Licenciatura em
Biologia, ofertados também por EAD, mas pela Universidade Federal do Vale do São
Francisco (UNIVASF).

Algumas adpatações no processo de replicação do estudo foram necessários, tais
como: mudança de tecnologia, ajuste nos scripts de coleta de dados e redução de
cinco para três algoritmos de classificação. Essas alterações não alteraram os
objetivos do trabalho, apenas forneceram novas e adequadas condições para o seu
desenvolvimento.

Assim, a principal questão a ser esclarecida neste trabalho é se um conjunto de
variáveis representativas da Teoria da Distância Transacional (TDT) e os alguns
dos algoritmos classificadores, também podem ser usados em modelos preditivos de
evasão na EAD, em um cenário diferente do originalmente apresentado por
\citeonline{ramos2016abordagem}.

Espera-se com este trabalho contribuir para o fortalecimento da EAD, além de
fomentar a linha de pesquisa voltada para o estudo das tecnologias educacionais,
tão evidenciadas e diversificadas, a partir do uso cada vez maior das
tecnologias de informação e comunicação no processo de ensino/aprendizagem,
particularmente aquelas destinadas a reduzir os atuais índices de evasão
verificados na modalidade.

\section{Objetivos}

Esta pesquisa será desenvolvida com o propósito de atingir os seguintes
objetivos geral e específicos:

\subsection{Objetivo geral}

Avaliar se um conjunto de modelos de predição desenvolvidos para outro cenário
de EAD, pode também ser usado para prever quais alunos têm tendência a evasão em
cursos nessa modalidade na UNIVASF, mantendo os resultados em níveis
satisfatórios, comparados aos originais.

\subsection{Objetivos específicos}
\begin{itemize}
  \item Adaptar os modelos preditivos já desenvolvidos para uma outra ferramenta
  tecnológica;
  \item Aplicar os classificadores em bases de dados de cursos EAD da UNIVASF;
  \item Avaliar os resultados dos classificadores segundo métricas consolidadas.
\end{itemize}

\section{Organização do texto}

Esse trabalho está organizado em 5 capítulos. No primeiro capítulo apresenta-se
o projeto, uma contextualização sobre o problema abordado, assim como os
objetivos gerais e específicos.

No segundo capítulo, é realiada uma revisão sobre a TDT, MDE e Aprendizagem
Supervisionada, com objetivo de promover um maior detalhamento sobre os
conceitos utilizados ao longo do texto. Também neste capítulo são apresentados
resumos de trabalhos relacionados com esta pesquisa.

O terceiro capítulo explora os detalhes da caracterização da pesquisa e a
metodologia aplicada, Descoberta de Conhecimento em Bases de Dados (KDD, do
inglês \textit{Knowledge Discovery in Databases}).

O quarto capítulo contém o cronograma de atividades para a disciplina Trabalho
de Conclusão de Curso II (TCC II).

E por fim, o quinto capítulo contém as considerações finais, os resultados
esperados e a contribuição da pesquisa.

		%--------------------------------------------------------------------------------------
% Este arquivo contém a sua funtamentação teórica
%--------------------------------------------------------------------------------------
\chapter{Referencial teórico}

\section{Seção de exemplo 1 - Como fazer citações}

Existem vários tipos de citações...


\section{Seção de exemplo 2 - Como inserir figuras}

Neste trabalho iremos exemplificar duas formas de se inserir figuras no Latex. O primeiro método insere, no documento, uma figura simples por meio do comando:

\textbackslash imagem\{ Escala \}\{ Arquivo sem extensão \}\{ Descrição \}\{ Fonte \}

\textbf{Obs.:} A fonte pode ser uma citação do tipo  \textbackslash citeonline\{\}.

A figura \ref{img:placeholder} é um exemplo deste método.

%--------------------------------------------------------------------------------------
% Esse é um exemplo de figura simples
%--------------------------------------------------------------------------------------
\imagem{0.15}{placeholder}{Uma figura simples}{O autor}

A figura \ref{img:figura1} é um exemplo do outro tipo de figura abordada aqui, chamada de figura composta. Esta figura é composta de outras subfiguras.
%--------------------------------------------------------------------------------------
% Esse é um exemplo de figura composta de outras subfiguras
%--------------------------------------------------------------------------------------
\begin{figure}[!htb]
\centering
    \caption{\label{img:figura1} Exemplo de figura composta}
    \subcaptionbox{\label{img:subfigura1} Subfigura 1}{\includegraphics[scale=.1]{img/placeholder}}\qquad
    \subcaptionbox{\label{img:subfigura2} Subfigura 2}{\includegraphics[scale=.1]{img/placeholder}}
    \vspace{1.5em}
    \legend{\textbf{Fonte:} \citeonline{SUA-REFERENCIA}}
\label{fig:dag}
\end{figure}

%\begin{figure}[!htb]
%\centering
%    \caption{\label{img:telas} Telas da aplicação cliente}
%    \subcaptionbox{\label{img:inicial} Abertura}{\includegraphics[scale=.12]{img/APP/inicial}}\qquad
%    \subcaptionbox{\label{img:login} \textit{Login}}{\includegraphics[scale=.12]{img/APP/login}}\qquad
%    \subcaptionbox{\label{img:cadastro} Cadastro}{\includegraphics[scale=.12]{img/APP/cadastro}}\qquad
%    \subcaptionbox{\label{img:hist-rel}Sobre}{\includegraphics[scale=.12]{img/APP/sobre}}\\
%    \vspace{1.5em}
%    \subcaptionbox{\label{img:dados_atuais}Dados atuais}{\includegraphics[scale=.15]{img/APP/atual}}\qquad
%    \subcaptionbox{\label{img:hist-time}Seleção de período}{\includegraphics[scale=.15]{img/APP/periodo}}\qquad
%    \subcaptionbox{\label{img:hist-rel}Exibir histórico}{\includegraphics[scale=.15]{img/APP/historico}}\\
%    \vspace{2.5em}
%    \legend{\textbf{Fonte:} O Autor}
%\label{fig:dag}
%\end{figure}

Para referenciar uma figura deve ser usada comando \textbackslash ref\{img:<label ou nome do arquivo>\}, como exemplo, estamos referenciando a figura \ref{img:placeholder}. Isso vale tanto para figuras simples quanto para as compostas, como por exemplo as figuras \ref{img:subfigura1} e \ref{img:subfigura2}. Ao inserir uma figura, ela é automaticamente identificada e incluída no elemento pré-textual da lista de figuras.




\section{Seção de exemplo 3 - Sobre tabelas}

As tabelas em Latex são deveras capciosas, por isso não serão abordadas em sua completude neste documento.

Há um site que possui uma ferramenta interessante para ser utilizada na construção tabelas em Latex.

\centerline{\href{https://www.tablesgenerator.com/}{ O Tables Generator } <-- Isto é um \textit{link} :D}

Contudo, busquem entendimento sobre o assunto, pois tabelas são elementos textuais importantes e enriquecem muito o texto, quando bem construídas.

A tabela \ref{tab:crossplatform} é um exemplo de como uma tabela pode ser construída, assim como a tabela do anexo \ref{anex:anexo1}.

\begin{table}[!htb]
	\centering
	\caption{\label{tab:crossplatform} Tipos de aplicações e abordagens preferenciais.}
	\begin{adjustbox}{max width=\textwidth}
		\begin{tabular}{@{} p{5cm} ccc @{}}
		\toprule
		\textbf{Código da Aplicação} & \textbf{Web} & \textbf{Híbrida} & \textbf{Interpretada / Compilação Cruzada} \\ \hline

		\textbf{Aplicações baseadas em dados providos por um servidor} &
			3 & 2 & 1
		\\ \hline

		\textbf{Aplicações independentes} & 1 & 2 & 3\\ \hline

		\textbf{Aplicações baseadas em sensores e processamento de dados no dispositivo} & 1 & 2 & 3\\ \hline

		\textbf{Aplicações baseadas em sensores e processamento de dados no servidor} & 1 & 3 & 2\\ \hline

		\textbf{Aplicações Cliente-Servidor} & 1 & 3 & 2 \\ \bottomrule
	\end{tabular}
	\end{adjustbox}
	\legend{\textbf{Fonte:} \citeonline{raj2012study} (Traduzido)}
\end{table}

Também é possível criar quadros, que são ligeiramente diferente de tabelas. Acompanhe o exemplo no Quadro \ref{qua:confusionmatrix}

\begin{quadro}
	\centering
	\caption{\label{qua:confusionmatrix}Exemplo de matriz de confusão}
	\begin{tabular}{ll|c|c|}
		\cline{3-4}
		\multicolumn{1}{c}{\textbf{}} & \multicolumn{1}{c|}{\textbf{}} & \multicolumn{2}{l|}{\textbf{Classe prevista}} \\ \cline{3-4}
		 & \multicolumn{1}{c|}{\textbf{}} & Classe = 1 & Classe = 0 \\ \hline
		\multicolumn{1}{|l|}{\multirow{2}{*}{\textbf{Classe real}}} & Classe = 1 & $f_{11}$ & $f_{10}$ \\ \cline{2-4}
		\multicolumn{1}{|l|}{} & Classe = 0 & $f_{01}$ & $f_{00}$ \\ \hline
	\end{tabular}
	\Ididthis
\end{quadro}

\section{Subseção de exemplo 4 - Seções}
		\chapter{Metodologia proposta}

\section{Caracterização da pesquisa}

Segundo \citeonline{marconi2003fundamentos}, a pesquisa é um procedimento
formal, com método de pensamento reflexivo, que requer um tratamento científico
e se constitui no caminho para conhecer a realidade ou para descobrir verdades
parciais. A pesquisa é um procedimento sistemático e crítico,  que permite
descobrir novos fatos, relações ou leis acerca de qualquer campo do
conhecimento.

Uma pesquisa pode ser caracterizada segundo os seguintes critérios
\cite{gil2008metodos}:
\begin{enumerate}[label=\alph*)]
  \item Quanto à natureza: básica ou aplicada;
  \item Quanto aos objetivos: exploratória, descritiva ou explicativa;
  \item Quanto à abordagem: qualitativa ou quantitativa;
  \item Quanto aos procedimentos: documental, bibliográfica, experimental,
  levantamento, estudo de caso, entre outros.
\end{enumerate}

Este trabalho pode ser classificado como de natureza aplicada, já que será
aplicada uma metodologia de busca de conhecimentos em bancos de dados e métodos
de classificação para prever a evasão de cursos EAD.

Em relação aos objetivos podemos classificar este trabalho como pesquisa
exploratória e descritiva. Tendo como base \citeonline{gil2002elaborar}, a
pesquisa exploratória busca ampliar o conhecimento sobre o problema, procurando
torná-lo mais explícito ou a construção de hipóteses, tendo como objetivo
central o aperfeiçoamento de ideias ou a revelação de intuições. E a pesquisa
descritiva objetiva descrever características de determinado fenômeno ou
população. Este trabalho utiliza uma metodologia de exploração de conhecimento
para tentar prever um comportamento em um conjunto de uma população.

Quanto à abordagem, este trabalho é classificado como quantitativo, em razão da
utilização de abordagens algorítmicas de Mineração de Dados, a partir das quais
serão extraídas as características dos estudantes de EAD e aplicadas modelos de
classificação que farão a devida categorização.

No quesito procedimentos classificamos este trabalho como pesquisa experimental.
De acordo com \citeonline{gil2002elaborar}, a pesquisa experimental consiste em
determinar um objeto de estudo, selecionar as variáveis que seriam capazes de
influenciá-lo, definir as formas de controle e de observação dos efeitos  qua
variável produz no objeto.

No caso deste trabalho, o objeto de estudo é a evasão na EAD da UNIVASF e as
variáveis foram definidas pela TDT.

\section{Método}

Para tratamento e preparação dos dados para os diferentes modelos de
classificação que serão avaliados, será utilizado o processo de Descobrimento de
Conhecimento em Banco de Dados (do inglês, \textit{Knowledge Discovery in
Databases}, KDD) como descrito por \citeonline{tan2009introduccao} e ilustrado
na Figura \ref{reducedKdd}. Este processo envolve uma série de passos com o
objetivo de transformar dados brutos em informações úteis.

\imagem{.70}{reduced_kdd}{\label{reducedKdd}Fluxo básico do processo KDD}{\citeonline{tan2009introduccao}}

A fase de entrada de dados será desenvolvida, baseado no trabalho de
\citeonline{ramos2016abordagem}, coletando as variáveis mais relevantes que
poderiam representar cada um dos três construtos da TDT. Os dados de onde serão
retiradas as variáveis estão armazenados nas bases de dados do Moodle,
atualmente em uso pelos cursos de graduação oferecidos pela UNIVASF na
modalidade EAD.

Na etapa de pré-processamento de dados ocorrem as transformações e adaptações
dos dados para os algoritmos de Mineração de Dados. Entre essas transformações
podemos citar: normalização, limpeza de valores faltantes, identificação de
outliers, entre outros. Esta etapa, geralmente, exige muito tempo e esforço. A
correta execução deste passo resultará em melhores resultados nas etapas
posteriores.

No contexto deste trabalho, utilizaremos ferramentas de análises exploratórias e
\textit{scripts} de buscas em bancos de dados para construção da base de dados a
ser utilizada na etapa posterior.

Em seguida, ocorre a etapa de mineração de dados, onde são buscados padrões de
interesse ou características que representem as tendências dos dados, entre os
métodos de busca de padrões podemos citar: clusterização, classificação,
regressão, entre outros.

Para este trabalho, utilizaremos os algoritmos de classificação a seguir: Árvore
de Decisão, KNN, e Regressão Logística. Nesta etapa, os parâmetros dos
algoritmos de classificação serão ajustados para que a performance dos mesmos
seja melhorada.

Na última etapa, pós-processamento, serão avaliados e interpretados os padrões
extraídos na etapa de mineração, podem ocorrer retornos a qualquer etapa
anterior para mais iterações. Esta etapa pode envolver a visualização dos
padrões e modelos gerados, ou visualização dos dados fornecidos. Neste passo, o
conhecimento descoberto será documentado para possível uso posterior, em uma
ferramenta de geração de relatórios ou de visualização, tipo dashboard.

\section{Materiais}

\subsection{Moodle}

O Moodle\footnote{\url{https://moodle.org/} Acesso em: 06 de mar. 2019} é uma
plataforma de ensino projetada para oferecer a educadores, administradores e
estudantes, com uma sistema integrado, simple e robusto, a criação de ambientes
de aprendizado personalizados. É financiado por uma rede de mais de 80 empresas
ao redor do mundo.

Moodle é um software de código aberto sob a licença GNU General Public License.
Qualquer pessoa pode adaptá-lo, estendê-lo ou modificá-lo, tanto para uso
comercial ou não-comercial, sem nenhum tipo taxa de licenciamento e se
beneficiando de sua eficiência e custo, flexibilidade e outras vantagens de usar
o Moodle.

\subsection{MySQL}

MySQL\footnote{\url{https://www.mysql.com/} Acesso em: 06 de mar. 2019} é a base
de dados mais popular no mundo. Provê performance, confiabilidade e facilidade
de uso, MySQL vem liderando a escolha de aplicações web, usado por grandes
empresas na internet como: Facebook, Twitter, YouTube, Yahoo! e muitas outras.

MySQL é sistema de gerenciamento de banco de dados (SGDB), baseado na linguagem
SQL (do inglês, Structured Query Language). Entre as vantagens suas vantagens
podemos listar: portabilidade, compatibilidade, excelente desempenho e
estabilidade, facilidade de manuseio e é um software livre sob a licença GPL.

\subsection{Python}

Python\footnote{\url{https://www.python.org/} Acesso em: 06 de mar. 2019} é uma
linguagem de programação de código aberto classificada como linguagem de alto
nível de abstração. Considerada de fácil manuseio mesmo por usuários iniciantes.
É mantida e desenvolvida pela Python Software Foundation

Graças a sua enorme comunidade, existem diversos pacotes e bibliotecas
desenvolvidas em Python para as mais variadas tarefas, desde servidores HTTP,
desenvolvimento de aplicações desktop até mineração de dados, inteligência
artificial e estatística.

\subsection{Anaconda Python Distribution}

A distribuição de código aberto
Anaconda\footnote{\url{https://www.anaconda.com/} Acesso em: 06 de mar. 2019}  é
uma maneira fácil de realizar tarefas de mineração de dados e aprendizado de
máquina em ambientes Linux, Windows ou Mac OS X. Anaconda é um gerenciador de
pacotes e ambientes e uma distribuição Python especializada em data science com
mais de 1500 pacotes de código aberto.

\subsection{Jupyter Notebook}

Jupyter Notebook\footnote{\url{https://jupyter.org/} Acesso em: 06 de mar. 2019}
é uma aplicação web de código aberto que permite a criação e compartilhamento de
documentos que contém código em tempo de execução, equações, visualizações e
textos narrativos. Funciona como uma IDE (do inglês, Integrated Development
Environment) e foi desenvolvido para tarefas de limpeza e transformação de
dados, simulações numéricas, modelagem estatística, visualização de dados,
aprendizado de máquina e mais.

Jupyter Notebook suporta mais de 40 linguagens de programação incluindo Python e
já vem pré configurado na distribuição Anaconda.

\subsection{Python Data Analysis Library}

Python Data Analysis Library\footnote{\url{https://pandas.pydata.org/} Acesso
em: 06 de mar. 2019}, ou simplesmente pandas, é uma biblioteca de código aberto
sob a licença BSD que provê estruturas de dados e ferramentas de análise de
dados de alta performance e fácil uso para a linguagem de programação Python.
Pandas proporciona estruturas de dados rápidas, flexíveis e expressivas
desenvolvidas para uso com dados relacionais ou etiquetados.

A biblioteca pandas já vem configurada para uso na distribuição Anaconda.

\subsection{Numpy}

NumPy\footnote{\url{http://www.numpy.org/} Acesso em: 06 de mar. 2019} é o
pacote fundamental para computação científica em Python. Contento, além de outra
funcionalidades, um poderoso vetor n-dimensional, funções de broadcast
sofisticadas, ferramentas de integração com códigos C/C++ e Fortran, ferramentas
de algebra linear, transformadas de Fourier e números aleatórios.

Além dos óbvios usos científicos, NumPy também pode ser usado como um invólucro
para dados genéricos. Tipos de dados arbitrários podem ser definidos, isso
permite que  seja integrado de forma rápida com uma miríade de bases de dados.

NumPy é uma biblioteca de código aberto sob a licença BSD e é presente na
distribuição Anaconda.

\subsection{Scikit-learn}

Scikit-learn\footnote{\url{https://scikit-learn.org/} Acesso em: 06 de mar.
2019} é um módulo Python para aprendizado de máquina de código aberto sob a
licença BSD. Além das principais tarefas de mineração, como: classificação,
regressão e clusterização a biblioteca proporciona as visualizações mais básicas
para análise exploratória. Scikit-learn é compatível com pandas e NumPy e pode
ser encontrado na distribuição Anaconda.

		\chapter{Resultados} \label{ch:RD}



\section{Seção de exemplo 1 - Códigos} \label{sec:resex1}

\subsection{Subseção de exemplo 1 - Inserindo trechos de códigos}
 
O nosso querido Leonardo Cavalcante providenciou um comando que deixa nossos trechos de códigos bonitinhos e gera um elemento pré-textual de Lista de Códigos. 

Os códigos são adicionados através do comando seguinte:

\textbackslash sourcecode\{ Descrição \}\{Label\}\{Linguagem\}\{Arquivo com extensão\}

Um exemplo pode ser visto no código \ref{cmd:cron} abaixo.

\sourcecode{Configuração do intervalo de execução no Script Agendador}{cron}{javascript}{cron.js}


\section{Seção de exemplo 2 - Listas} \label{sec:resex2}

\subsection{Subseção de exemplo 2 - Lista de itens} 

Existem alguns tipos de listas no Latex, iremos exemplificar a lista sem numeração (seção \ref{subsubsec:itemize}), a lista enumerada (seção \ref{subsubsec:enumerate}) e a lista mista (seção \ref{subsubsec:mista}). As listas podem ser encadeadas de diversas maneiras,
de acordo com a necessidade do autor.

\subsubsection{Subsubseção de exemplo 1 - Lista sem numeração} \label{subsubsec:itemize}

Este é um exemplo de lista sem numeração.

\begin{itemize}
	\item \textbf{Cadastrar usuário}

		\begin{itemize}
    		\item Atores
		    	\begin{itemize}
    		    	\item Usuário
		    	\end{itemize}

	    	\item Fluxo de eventos primário
			    \begin{itemize}
	    		    \item o usuário deve se cadastrar informando seu nome, \textit{e-mail} e senha;
		        	\item a API armazena os dados do usuário;
		    	    \item o usuário é liberado para realizar o \textit{login}.
			    \end{itemize}

    		\item Fluxo alternativo
			    \begin{itemize}
		    	   \item o usuário desiste de se cadastrar e cancela o caso de uso clicando no botão voltar.
	    		\end{itemize}

		\end{itemize}
	
\end{itemize}

\subsubsection{Subsubseção de exemplo 2 - Lista enumerada} \label{subsubsec:enumerate}

Este é um exemplo de lista enumerada.

\begin{enumerate}
	\item O Usuário deseja ver o histórico das variáveis climáticas, então através da interface de usuário escolhe o período ao qual o histórico se refere;
	\item A aplicação solicita à API através de uma requisição HTTP contendo o momento de início e o momento do fim do período em seus parâmetros;     			\item A API recebe a solicitação e se comunica com a base de dados, então requere as informações quem possuem a data de leitura no intervalo escolhido;
	\item A base de dados retorna os dados em formato Json para a API;
	\item A API responde à requisição retornando os dados, também em formato Json, para a aplicação cliente;
	\item A aplicação cliente renderiza os gráficos utilizando o conjunto de dados obtidos.
\end{enumerate}

\subsubsection{Subsubseção de exemplo 3 - Lista mista} \label{subsubsec:mista}

Este é um exemplo de lista mista.

\begin{itemize}
	\item \textbf{Cadastrar usuário}

		\begin{itemize}
    		\item Atores
		    	\begin{itemize}
    		    	\item Usuário
		    	\end{itemize}

	    	\item Fluxo de eventos primário
			    \begin{enumerate}
	    		    \item o usuário deve se cadastrar informando seu nome, \textit{e-mail} e senha;
		        	\item a API armazena os dados do usuário;
		    	    \item o usuário é liberado para realizar o \textit{login}.
			    \end{enumerate}

    		\item Fluxo alternativo
			    \begin{itemize}
		    	   \item o usuário desiste de se cadastrar e cancela o caso de uso clicando no botão voltar.
	    		\end{itemize}

		\end{itemize}

	\item \textbf{Visualizar dados atuais}

		\begin{itemize}
		    \item Atores
	    		\begin{itemize}
		    	    \item Usuário
			    \end{itemize}
    
	    	\item Pré-condições
			    \begin{itemize}
		     	   \item o usuário deve estar autenticado
			    \end{itemize}

	    	\item Fluxo de eventos primário
			    \begin{enumerate}
		    	    \item o usuário deve efetuar o \textit{login} informando o \textit{e-mail} e a senha;
	    		    \item caso o usuário não seja autenticado, o sistema informa a respeito de credenciais inválidas e encerra o caso de uso;
		    	    \item a API autentica o usuário;
    			    \item o usuário é liberado para visualizar os dados atuais dos sensores da estação;
		        	\item após a visualização o usuário pode finalizar o caso de uso ou efetuar uma nova consulta se desejar.
			    \end{enumerate}

    		\item Fluxo alternativo
			    \begin{itemize}
    			   \item o usuário desiste de visualizar os dados atuais e cancela o caso de uso clicando no botão voltar.
			    \end{itemize}

		\end{itemize}

	\item \textbf{Visualizar histórico}

		\begin{itemize}
		    \item Atores
	    		\begin{itemize}
		    	    \item Usuário
	    		\end{itemize}

	    	\item Pré-condições
    			\begin{itemize}
			        \item o usuário deve estar autenticado
			    \end{itemize}

		    \item Fluxo de eventos primário
			    \begin{enumerate}
			        \item o usuário deve efetuar o \textit{login} informando o \textit{e-mail} e a senha;
			        \item caso o usuário não seja autenticado, o sistema informa a respeito de credenciais inválidas e encerra o caso de uso;
			        \item a API autentica o usuário;
			        \item o usuário é liberado para escolher qual período cujo histórico será exibido;
			        \item o usuário seleciona as variáveis a serem exibidas no gráficos de linhas;
			        \item após a visualização do histórico o usuário pode finalizar o caso de uso se desejar.
			    \end{enumerate}

		    \item Fluxo alternativo
			    \begin{enumerate}
			        \item após a escolha do período de exibição do histórico o usuário pode voltar para a tela anterior e escolher um novo período;
			        \item o histórico é exibido para o usuário;
			        \item após a visualização do histórico o usuário pode finalizar o caso de uso ou efetuar uma nova consulta se desejar.
			    \end{enumerate}

		    \item Fluxo alternativo
			    \begin{enumerate}
			        \item o usuário desiste de visualizar o histórico e cancela o caso de uso clicando no botão voltar.
			    \end{enumerate}
		\end{itemize}
\end{itemize}
		\input{tex/conclusao}

	\postextual
		\bibliography{tex/references}
		\begin{anexosenv}
\chapter{Comandos seriais da estação meteorológica \textit{Vantage Vue™}} \label{anex:anexo1}

\begin{center}
\scalefont{0.85}
\begin{longtable}{ll}
\caption{Comandos seriais suportados pela estação meteorológica \textit{Vantage Vue™}}\\
\hline
\multicolumn{1}{c}{\textbf{Instrução}} & \multicolumn{1}{c}{\textbf{Descrição}} \\ \hline
\endfirsthead

\multicolumn{2}{c}%
{{\bfseries \tablename\ \thetable{} -- Continuação da página anterior}} \\

\hline
\multicolumn{1}{c}{\textbf{Instrução}} & \multicolumn{1}{c}{\textbf{Descrição}} \\ \hline
\endhead

\multicolumn{2}{r}{{Continua na próxima página}} \\
\endfoot

\endlastfoot

\multicolumn{2}{c}{\cellcolor{gray!25}\textbf{Comandos de teste}}                                                   		 \\ \hline
\textbf{TESTE}                            & Envia a \textit{string} "TEST\textbackslash n" de volta  \\ \hline
\textbf{WRD}                        & Responde com o tipo de estação meteorológica \\ \hline
\textbf{RXCHECK}                        & Responde com o diagnóstico do Console \\ \hline
\textbf{RXTEST}                       & Muda a tela do console de \textit{"Receiving from"} para tela de dados atuais                                                        \\ \hline
\textbf{VER}                           & Responde com a data do \textit{firmware}                                                             \\ \hline
\textbf{RECEIVERS}                    & Responde com a lista das estações que o console "enxerga" \\ \hline
\textbf{NVER}                       & Responde com a versão do \textit{firmware}                                                             \\ \hline
\multicolumn{2}{c}{\cellcolor{gray!25}\textbf{Comandos de dados atuais}}                                             \\ \hline
\textbf{LOOP}                     & Responde com a quantidade de pacotes especificada a cada 2s        \\ \hline
\textbf{LPS}                & Responde a cada 2s com a quantidade de pacotes diferentes especificada          \\ \hline
\textbf{HILOWS}                & Responde com todo os dados de \textit{high/low}                 \\ \hline
\textbf{PUTRAIN}                      & Seta a quantidade anual de precipitação \\ \hline
\textbf{PUTET}                 & Seta a quantidade anual de evapotranspiração        \\ \hline
\multicolumn{2}{c}{\cellcolor{gray!25}\textbf{Comandos de \textit{download}}}                                     		 \\ \hline
\textbf{DMP}                 & Faz o \textit{download} de todo o arquivo de memória \\ \hline
\textbf{DMAFT}                   & Faz o \textit{download} de todo o arquivo de memória após a data especificada \\ \hline
\multicolumn{2}{c}{\cellcolor{gray!25}\textbf{Comandos da EEPROM}}                                     		 \\ \hline
\textbf{GETEE}                 & Lê toda a memória EEPROM \\ \hline
\textbf{EEWR}                   & Escreve um \textit{byte} de dados à partir do endereço especificado                                   \\ \hline
\textbf{EERD}                   & Lê a quantidade de dados especificada iniciando no endereço especificado                                   \\ \hline
\textbf{EEBWR}                   & Escreve os dados na EEPROM                                    \\ \hline
\textbf{EEBRD}                   & Lê os dados da EEPROM \\ \hline
\multicolumn{2}{c}{\cellcolor{gray!25}\textbf{Comandos de calibração}}                                     		 \\ \hline
\textbf{CALED}                 & Envia os dados da temperatura e umidade corrente para atribuir à calibração \\ \hline
\textbf{CALFIX}                   & Atualiza o \textit{display} quando os números de calibração mudam\\ \hline
\textbf{BAR}                   & Seta os valores da elevação e o \textit{offset} do barômetro quando a localização é alterada                                   \\ \hline
\textbf{BARDATA}                   & Mostra os valores atuais da calibração do barômetro                                   \\ \hline \\
\multicolumn{2}{c}{\cellcolor{gray!25}\textbf{Comandos de limpeza}}                                     		 \\ \hline
\textbf{CLRLOG}                 & Limpa todo o arquivo de dados                                                       \\ \hline
\textbf{CLRALM}                   & Limpa todos os limiares dos alarmes                                   \\ \hline
\textbf{CLRCAL}                   & Limpa todos os \textit{offsets} da calibração da temperatura e da umidade \\ \hline
\textbf{CLRGRA}                   & Limpa o gráfico do console \\ \hline
\textbf{CLRVAR}                   & Limpa o valor da precipitação ou da evapotranspiração \\ \hline
\textbf{CLRHIGHS}                   & Limpa todos os valores de pico diários, mensais ou anuais                                   \\ \hline
\textbf{CLRLOWS}                   & Limpa todos os valores de mínimos diários, mensais ou anuais \\ \hline
\textbf{CLRBITS}                   & Limpa os \textit{bits} de alarme ativos                                  \\ \hline
\textbf{CLRDATA}                   & Limpa todos os dados atuais                                   \\ \hline
\multicolumn{2}{c}{\cellcolor{gray!25}\textbf{Comandos de configuração}}                                     		 \\ \hline
\textbf{BAUD}                 & Atribui o valor do \textit{baudrate} do console                                                       \\ \hline
\textbf{SETTIME}                   & Define a data e a hora do console                                   \\ \hline
\textbf{GAIN}                   & Define o ganho do receptor de rádio                                   \\ \hline
\textbf{GETTIME}                   & Retorna a hora e a data atual do console                                   \\ \hline
\textbf{SETPER}                   & Define o intervalo de arquivamento                                   \\ \hline
\textbf{STOP}                   & Desabilita a criação dos registros                                   \\ \hline
\textbf{START}                   & Habilita a criação dos arquivos \\ \hline
\textbf{NEWSETUP}                   & Reinicia o console após alguma configuração nova                                  \\ \hline
\textbf{LAMPS}                   & Liga ou desliga as lâmpadas do console \\ \hline

%\label{tab:6}
\end{longtable}
\fonte{\citeonline{VSPDOC} (Traduzido).}
\end{center}


\end{anexosenv}


\end{document}
